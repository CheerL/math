\documentclass[UTF8]{ctexart}
%\documentclass{article}

\title{泛函分析第 次作业}
\author{林陈冉}
\date{\today}
\usepackage{geometry}
\usepackage{amsmath}
\usepackage{amssymb}

\geometry{papersize={21cm,29.7cm}}
\geometry{left=1.91cm,right=1.91cm,top=2.54cm,bottom=2.54cm}

\begin{document}
\maketitle
\paragraph{5.26}(1) $\forall v \in H$ , 
$v=\sum^{\infty}_{n=1}(e_n,v)$ , 
由 $v \in H$ , 
可知 $\vert v\vert^2=\sum^{\infty}_{n=1}\vert (e_n,v)\vert^2 < \infty$ , 
则 $\lim_{n\rightarrow\infty}\vert (e_n,v)\vert^2 = 0$ , 
自然的, $\lim_{n\rightarrow\infty}(e_n,v) = 0$ , 
故 $(e_n,v)\rightharpoonup 0$ .

\quad

\quad(2) ${a_n}$ 有界, 
设 $\vert a_n\vert < M$ , 那么有
\[\vert u_n\vert^2 = \vert (\frac{1}{n}\sum^{\infty}_{i=1}a_ie_i,\frac{1}{n}\sum^{\infty}_{i=1}a_ie_i)\vert = \frac{1}{n^2}\sum^{\infty}_{n=1}a_n^2 < \frac{1}{n}M^2\]
则 $\lim_{n \rightarrow \infty}\vert u_n\vert^2 < \lim_{n \rightarrow \infty}\frac{1}{n}M^2 = 0$ ,
故 $\vert u_n\vert \rightarrow 0$ .

\quad

\quad(3) 

\paragraph{5.28}(1) $H$ 是可分的, $V \subset H$, 则显然 $V$ 也是可分的.
记 $\{v_n\}$ 是 $V$ 的一个可数稠密子集, 
记 $\{v_1, v_2, \cdots, v_n\}$ 张成的空间为 $V_n$ , 则 $\bigcup^\infty_{n=1}V_n$ 在 $V$ 中是稠密的, 由 $V$在 $H$ 中稠密, 可知 $\bigcup^\infty_{n=1}V_n$ 在 $H$ 中也是稠密的.

\quad对于 $V_1$ , 可以任意找一个单位向量 $e_1$ ; 
对于 $V_2$ , 当 $V_1\neq V_2$ , 
因为 $V_2$ 是有限维的, 可以找另一个单位向量 $e_2$ , 
使 $\{e_1, e_2\}$ 是 $V_2$ 的一组正交基;
当 $V_1= V_2$ , 则直接考虑 $V_3$ .

\quad对所有的 $V_n$ 重复这样的操作, 可以得到 $H$ 的一组正交基 $\{e_1, e_2, \cdots, e_n, \cdots \}$ , 显然它是属于 $V$ 的.

\quad

\quad(2) 由 $H$ 是可分的, 则存在 $H$ 的一个可数稠密子集 $\{v_k\}$ ,
那么  $\{v_k\}\bigcup\{e_n\}$ 也是 $H$ 的一个可数稠密子集.
记 $\{e_1, e_2, \cdots, e_n, \cdots \}$ 张成的空间为 $V_0$ , 
$\{e_1, e_2, \cdots, e_n, \cdots \}\bigcup\{v_1, v_2, \cdots, v_k\}$ 张成的空间为 $V_k(k>0)$ , $\bigcup^\infty_{n=1}V_n$ 在 $H$ 中是稠密的 .

\quad对于 $V_1$ , 当 $V_1\neq V_0$ , 令 
\[u_1=\frac{P_{V_0}v_1}{\vert P_{V_0}v_1\vert}\]
由 $V_0$ 是 $\bigcup^\infty_{n=1}\{e_n\}$ 张成的, 可知它是 $H$ 的闭凸子空间, 
则有 $\forall x\in V_0$, $(x, u_1)=0$ , 故 $ (e_n,u_i)=0$.
且显然 $\vert u_1\vert=1$ , 
故 $\{u_1, e_1, e_2, \cdots, e_n, \cdots \}$ 是 $V_1$ 的一组正交基.
当 $V_1 = V_0$ , 则直接考虑 $V_2$ 与 $V_1$ , 以完全相同的办法可以找到 $u_2$ 构成 $V_2$ 的一组正交基.

\quad对所有的 $V_n$ 重复这样的操作, 可以到的 $H$ 的一组正交基 $\{u_1, u_2, \cdots, u_k, \cdots \}\bigcup\{e_1, e_2, \cdots, e_n, \cdots \}$ , 显然它是包含 $\{e_1, e_2, \cdots, e_n, \cdots \}$ 的.

\end{document}