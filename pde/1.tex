\documentclass[a4paper, UTF8]{ctexart}				%中文环境
%\documentclass{article}							%英文环境

%---------------------------宏包加载----------------------------
    \usepackage{amsmath}
    \usepackage{amssymb}
    \usepackage{amsthm}
    \usepackage{geometry}
        \geometry{left = 2.54cm, right = 2.54cm,
            top = 3.18cm, bottom = 3.18cm}			%页边距设置
    \usepackage{fancyhdr}
        \pagestyle{fancy}
        %\lfoot{\today}   							%左页脚
        \cfoot{\thepage}							%中页脚
        %\rfoot{林陈冉}	 						 		%右页脚
        \setlength{\parskip}{0.5 \baselineskip}		%段距
    \usepackage{hyperref}  							%打开超链接
        \hypersetup{colorlinks=false}				%取消超链接颜色
    \usepackage{tikz}
    \usepackage{multirow}

%---------------------------标题设置----------------------------
    \title{偏微分方程第一次作业}
    \author{林陈冉}
    \date{\today}

%---------------------------定理环境----------------------------
    \newtheorem{theo}{\bf 定理}[section]			  %新建定理环境, 标题为"定理", 以section为计数器标记
    \newtheorem{define}{\bf 定义}[section]
    \newtheorem{algo}{\bf 算法}
    \renewcommand{\proofname}{\bf 证明}			  %重命名定理环境, 标题为"证明"
    \numberwithin{equation}{section}				%以section为计数器标记公式

%-----------------------------正文------------------------------
\begin{document}									%开始正文
    \maketitle										%生成标题
    \paragraph{1}
        Weierstrass函数的例子是 
        \[
            f(x) = \sum^{\infty}_{0} a^n \cos (b^n \pi x)
        \]
        其中 $0 < a < 1$ , $b > 0$ 且为奇数, $ab > 1 + \frac{3}{2}\pi$ .

        首先说明连续性. 设 $f_n(x) = a^n \cos (b^n x)$ , 对 $\forall x$ , $f_n(x) < a^n$ , 而 $\sum^{\infty}_{0} a^n < \infty$ , 则 $f(x) < \infty$, 可知这个级数一致收敛, 又由每一个 $f_n$ 连续可知 $f$ 连续.
        
        再说明不可导. (其实说明不来) 若 $f(x)$ 可导, 
        \[
            f'(x) = \sum^{\infty}_{n=0} f_n'(x) = \sum^{\infty}_{n=0} - (ab)^n \sin (b^n x)
        \]
        但 $ab > 1 + \frac{3}{2} \pi$ , 好像这个级数不收敛?

    \paragraph{2}
        Holder不等式等号成立的条件: 
        \begin{itemize}
            \item $u(x) = 0$ 或 $v(x)=0$;
            \item $\exists k > 0$ , $\vert{u(x)}\vert^p = k \vert{v(x)}\vert^q$ 几乎处处成立.
        \end{itemize}

        第一个情况显然成立, 考虑第二种情况. 从定理证明可知, 只要考虑什么时候 
        \[
            s_1 t_1 = \int^{t_1}_{0} t^{p - 1}dt + \int^{s_1}_{0} s^{q - 1}ds
        \]
        容易知道等号当且仅当 $t_1^p = s_1^q$ 时成立.
        
        若该式几乎处处成立, 则有 
        \[
            \int^{}_{\Omega}s_1(x) t_1(x)dx = \frac{\int^{}_{\Omega} t_1(x)^pdx}{p} + \frac{\int^{}_{\Omega} s_1(x)^qdx}{q} = 1
        \]
        若否则这个积分严格小于1. 又由 
        \[
            t_1(x) = \frac{\vert{u(x)}\vert}{(\int^{}_{\Omega} \vert{u(x)^p}\vert dx)^{\frac{1}{p}}} , 
            \quad 
            s_1(x) = \frac{\vert{v(x)}\vert}{(\int^{}_{\Omega} \vert{v(x)^q}\vert dx)^{\frac{1}{q}}}
        \]
        可得 
        \[
            \vert{u(x)}\vert^p = \frac{\int^{}_{\Omega} \vert{u(x)^p}\vert dx}{\int^{}_{\Omega} \vert{v(x)^q}\vert dx} \vert{v(x)}\vert^q
        \]
        上式等价于 $t_1^p = s_1^q$ , 是Holder不等式等号成立的情况2.

    \paragraph{3} Friedrichs不等式在无界区域下不成立的例子 
    \[
        f(x) = 1
    \]

\end{document}										%结束正文
