\documentclass[a4paper, UTF8]{ctexart}				%中文环境
%\documentclass{article}							%英文环境

%---------------------------宏包加载----------------------------
    \usepackage{amsmath}
    \usepackage{amssymb}
    \usepackage{amsthm}
    \usepackage{geometry}
        \geometry{left = 2.54cm, right = 2.54cm,
            top = 3.18cm, bottom = 3.18cm}			%页边距设置
    \usepackage{fancyhdr}
        \pagestyle{fancy}
        %\lfoot{\today}   							%左页脚
        \cfoot{\thepage}							%中页脚
        %\rfoot{林陈冉}	 						 		%右页脚
        \setlength{\parskip}{0.5 \baselineskip}		%段距
    \usepackage{hyperref}  							%打开超链接
        \hypersetup{colorlinks=false}				%取消超链接颜色
    \usepackage{tikz}
    \usepackage{multirow}

%---------------------------标题设置----------------------------
    \title{离散数学第2次作业}
    \author{林陈冉}
    \date{\today}

%---------------------------定理环境----------------------------
    \newtheorem{theo}{\bf 定理}[section]			  %新建定理环境, 标题为"定理", 以section为计数器标记
    \newtheorem{define}{\bf 定义}[section]
    \newtheorem{algo}{\bf 算法}
    \renewcommand{\proofname}{\bf 证明}			  %重命名定理环境, 标题为"证明"
    \numberwithin{equation}{section}				%以section为计数器标记公式

%-----------------------------正文------------------------------
\begin{document}									%开始正文
    \maketitle										%生成标题
    \paragraph{2.1.13} 
        证明过程中, 假设 $S = \{a, b, c, d, \cdots\}$ , 即至少有4条直线, 这暗含着要求在 $n = 3$ 的情况下命题成立, 证明过程只说明了 $n = 1, 2$ 时命题成立, 且事实上 $n = 3$ 时确实不成立.
    \paragraph{2.5.4}
        (a) $n^2 - 1 = (n+1)(n-1)$ , 当 $n$ 是奇数, 则 $n + 1, n-1$ 都是偶数, 
        故 $n^2 - 1$ 是 $2 \times 2 = 4$ 的倍数.

        (b) $n^3 - n = n(n + 1)(n - 1)$ , 对 $\forall n$ , $n, n + 1$ 中至少有一个偶数, $n, n + 1, n - 1$ 中有一个是3的倍数, 故 $n^3 - n$ 是6的倍数.
    \paragraph{2.5.5}
        记 $C = \{girls who like to paly chess\}$ , $S = \{girls who like to  play soccer\}$ , $B = \{girls who like biking\}$ .
        \begin{equation*}
            \begin{split}
                \vert{S \cup B \cup c}\vert
                & = \vert{S}\vert + \vert{B}\vert + \vert{C}\vert - \vert{S \cap B}\vert - \vert{A \cap C}\vert - \vert{B \cap C}\vert + \vert{S \cap B \cap C}\vert\\
                & = \vert{B}\vert + 23 + 18 - 9 - 7 - 12 + 4 
                  = \vert{B}\vert +17 = 40
            \end{split}
        \end{equation*}
        则容易求得喜欢骑行的有23人
    \paragraph{补充1} 
        正向直接求解即可, 实际上, $\xi \ge 1$ , $\eta \ge 2$ , $\zeta \ge 3$ , 故有 $1 + 2 + 3 = 6$ 组解.
    \paragraph{补充2} 
        设 $A = \{n=0\,mod\,5\}$ , $B = \{n=0\,mod\,6\}$ , $C = \{n=0\,mod\,8\}$ , 能被5或6或8整除的数构成集合 $A \cup B \cup C$
        \begin{equation}
            \begin{split}
                \vert{A \cup B \cup C}\vert
                & = \vert{A}\vert + \vert{B}\vert + \vert{C}\vert - \vert{A \cap B}\vert - \vert{A \cap B}\vert - \vert{B \cap C}\vert + \vert{A \cap B \cap C}\vert\\
                & = 200 + 166 + 125 - 33 - 25 - 20 + 4 = 412
            \end{split}
        \end{equation}
        故1到1000间不能被5和6和8整除的数共588个.
    \paragraph{补充3}
        (1) 两个z相邻有 $\frac{5!}{2!} = 60$ 种排列, 两个g相邻也有 $\frac{5!}{2!} = 60$ 种排列, 两个z和两个g同时相邻有 $4! = 24$ 种排列, 故两个z和两个g同时不相邻有 $\frac{6!}{2!2!} - 60 - 60 + 24 = 84$ 种排列.

        (2) 考虑先排列好i, a, g, g, 有 $\frac{4!}{2!} = 12$ 种排列. 再插入z, z, 故有 $6 \times 3 + 4 \times 1 + 2 \times 1 = 24$ 种排列
    \paragraph{补充4} 
        设 $A = \{n\text{是平方数}, 1 \le n \le 10000\}$ , $B = \{n\text{是立方数}, 1 \le n \le 10000\}$ . $\vert{A}\vert = \lfloor{10000^{\frac{1}{2}}}\rfloor = 100$ , $\vert{B}\vert = \lfloor{10000^{\frac{1}{3}}}\rfloor = 21$ , $\vert{A \cap B}\vert = \lfloor{10000^{\frac{1}{6}}}\rfloor = 4$ , 故是平方数或者立方数的数构成集合 $A \cup B$ 
        \[
            \vert{A \cup B}\vert = \vert{A}\vert + \vert{B}\vert - \vert{A \cap B}\vert = 117
        \]
        故1到10000间的平方数或立方数共117个.
    \paragraph{补充1} 
        假设命题不成立, 存在子集使任意两数的差大于等于 $3$ , 则最小数和最大数的差大于等于 $3n$ , 由于最小数至少是 $1$ , 则最大数至少是 $3n + 1$ , 但原集合最大数是 $3n$ , 产生矛盾. 故原命题成立.
    \paragraph{补充2} 
        可以更小, 只需要6个人, 直接证明这一点, 即可以证明原题. 转化考虑下面的问题: 房间中已经有 $k$ 个人, 再进入一个人, 使任意两组人年龄都不相等, 最多可以加到多少人?

        记第 $i$ 个进入房间的人年龄为 $a_i$ , 则至少 $a_k$ 不等于 $\{a_1, \cdots, a_{k-1}\}$ 的任意子集的和. 不妨认为大家按年龄顺序进入, 即 $i < j \Leftrightarrow a_i < a_j$ . 已知 $a_1 \ge 1$, 那么容易知道 $a_2 \ge 2$ . 
        
        来考虑所有人年龄都取到最小的情况, 即所有 $a_k$ 都取到等号, $a_1 = 1$ , $a_2 = 2$ . 那么当 $a_k = p$ , 说明 $\{a_1, \cdots, a_{k-1}\}$ 的子集可以组合出任意小于等于 $p - 1$ 的整数, 那么 $\{a_1, \cdots, a_{k-1}, a_k\}$ 的子集可以组合出任意小于等于 $2p - 1$ 的整数, 故 $a_{k+1} = 2p$ . 由 $a_1 = 1$ , 则 $a_k = 2^{k-1}$ , $\sum^{n}_{k = 1} a_k = 2^n - 1$ , $2^5 - 1 < 60 < 2^6 - 1$ , 即最多房间中可以有5人, $\{1, 2, 4, 8, 16\}$ 是满足题意的年龄集合. 那么当有6人时, 房间中必有两组人的年龄之和相等, 原题得证.
    \paragraph{补充3} 
        考虑这个子集集合中元素最少的子集 $A$ . 显然 $\vert{A}\vert \ge 1$ , 若 $A = \{a\}$ , 则由题意, 所有子集都必须含有元素 $a$ , 那么这样的子集集合共有子集 $2^{n-1}$ 个. 若 $\vert{A}\vert > 1$ , 则显然子集集合中的集合个数更少了, 故命题成立.
    \paragraph{补充4} 
        假设任意两点距离大于等于1, 考虑10个任意两点间距离大于等于1的点组成的最小面积, 这等价于10个半径为 $\frac{1}{2}$ 的互不相交的圆的圆心组成的最小面积.
        
        为了面积更小, 我们让圆尽可能相切, 容易证明一个圆最多和6个圆相切, 此时圆心之间连线也构成等边三角形.10个点最少构成9个边长为1的等边三角形, 面积为 $\frac{9\sqrt{3}}{2}$ , 但这刚好是边长为3的等边三角形的面积, 与点在三角形内矛盾, 故三角形内不可能有10个任意两点间距离大于等于1的点.
    \paragraph{补充5}
        记这任意取的两个数为 $a$ , $b$, 不妨认为 $a > b$ , $a = a' \, mod \, 100$ , $b = b' \, mod \, 100$
        $$  
            a + b = 0 \, mod \, 100 \Leftrightarrow a' + b' = 100 \quad
            a - b = 0 \, mod \, 100 \Leftrightarrow a' = b'
        $$
        则原题等价于: 在 $[0, 99]$ 中可以重复地取52个整数, 至少有2个整数和为100或相等. 
        
        上述命题的否定形式为: 在 $[0, 99]$ 中不重复地取52个整数, 任意2个整数的和都不等于100. 这显然不成立, 故原命题成立.
    \paragraph{补充6} 
        类似补充5, 记这任意取的三个数为 $a$ , $b$, $c$ 不妨认为 $a > b > c$ , $a = a' \, mod \, 3$ , $b = b' \, mod \, 3$ , $c = c' \, mod \, 3$ .
        $$a + b + c = 0 \, mod \, 3 \Leftrightarrow a' + b' + c' = 0, 3, 6$$
        则原题等价于: 在 $[0,2]$ 中可以重复的取5个整数, 任意3个整数的和被3整除. 其否定形式等价于: 在 $[0,2]$ 中可以重复的取5个整数构成集合$S$, $\{0, 0, 0\}$ , $\{0, 1, 2\}$ , $\{2, 2, 2\}$ 不是集合$S$的子集.
\end{document}										%结束正文
