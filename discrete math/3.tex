\documentclass[a4paper, UTF8]{ctexart}				%中文环境
%\documentclass{article}							%英文环境

%---------------------------宏包加载----------------------------
    \usepackage{amsmath}
    \usepackage{amssymb}
    \usepackage{amsthm}
    \usepackage{geometry}
        \geometry{left = 2.54cm, right = 2.54cm,
            top = 3.18cm, bottom = 3.18cm}			%页边距设置
    \usepackage{fancyhdr}
        \pagestyle{fancy}
        %\lfoot{\today}   							%左页脚
        \cfoot{\thepage}							%中页脚
        %\rfoot{林陈冉}	 						 		%右页脚
        \setlength{\parskip}{0.5 \baselineskip}		%段距
    \usepackage{hyperref}  							%打开超链接
        \hypersetup{colorlinks=false}				%取消超链接颜色
    \usepackage{tikz}
    \usepackage{multirow}

%---------------------------标题设置----------------------------
    \title{离散数学第3周作业}
    \author{林陈冉}
    \date{\today}

%---------------------------定理环境----------------------------
    \newtheorem{theo}{\bf 定理}[section]			  %新建定理环境, 标题为"定理", 以section为计数器标记
    \newtheorem{define}{\bf 定义}[section]
    \newtheorem{algo}{\bf 算法}
    \renewcommand{\proofname}{\bf 证明}			  %重命名定理环境, 标题为"证明"
    \numberwithin{equation}{section}				%以section为计数器标记公式

%-----------------------------正文------------------------------
\begin{document}									%开始正文
    \maketitle										%生成标题
    \paragraph{3.4.1}\quad 
        当 $n<2k$ , 这显然是不可能满足的. 当 $n \ge 2k$ , 可以认为是把 $n-2k$ 便士分给 $k$ 个孩子, 故分法数为 
        \[
            {n-k-1 \choose k-1}
        \]
    \paragraph{3.8.11}\quad 
        \[
            {{n} \choose {k}} = {{n-1} \choose {k}} + {{n-1} \choose {k-1}}
        \]
        由这个公式, Pascal三角形下一行的元素一定比上一行大.
    \paragraph{补充1}\quad 
        全排列总数为 $\frac{9!}{2!2!3!} = 15120$ 种. 8排列可以看成先去掉A, D, R, E, S 中的某一个再进行全排列, 总数为 $2 \times \frac{8!}{2!2!3!} + 2 \times \frac{8!}{2!3!} + \frac{8!}{2!2!2!} = 15120$ 种.
    \paragraph{补充2}\quad 
        由题可知, 每层隔板放的书不超过 $n$ 本. 记三层隔板中书的本数为 $n_1, n_2, n_3$, 那么有 $0 \le n_1 \le n$ , $n + 1 - n_1 \le n_2 \le n$ , $n_3 = 2n + 1 - n_1 - n_2$ , 故总放法有 
        \[
            \sum^{n}_{n_1=1} \; \sum^{n}_{n_2=n+1-n_1}
            {2n + 1 \choose n_1}{2n + 1 - n_1 \choose n_2} 
            = \sum^{n}_{n_1=1} \; \sum^{n}_{n_2=n+1-n_1}
            {2n+1 \choose {n_1 \, n_2 \, n_3}}
        \]
        
    \paragraph{补充3}\quad 
        我们知道, ${n \choose k}^2 = {n \choose n - k}^2$ , 那么当 $n = 2m-1$ 
        \[
            \sum^{n}_{k=0} (-1)^k {n \choose k}^2 
            = \sum^{m}_{k=0} \biggl((-1)^k {n \choose k}^2 + (-1)^{n-k} {n \choose n-k}^2\biggl) = 0
        \]

        那么当 $n = 2m$ 
        \[
            \sum^{n}_{k=0} (-1)^k {n \choose k}^2 
            = \sum^{m}_{k=0} \biggl((-1)^k {n \choose k}^2 + (-1)^{n-k} {n \choose n-k}^2\biggl) + (-1)^m {2m \choose m} = (-1)^m {2m \choose m}
        \]
        
        故综上有 
        \[
            \sum^{n}_{k=0} (-1)^k {n \choose k}^2 =
            \begin{cases}
                0, &\text{ if }n=2m-1\\
                (-1)^m {2m \choose m}, &\text{ if }n=2m
            \end{cases}
        \]
        

    \paragraph{补充4}\quad 
        令 $n = 3,4,5$ , 可得方程组 
        \[
            \begin{cases}
                3^3 = a + 3 b + 6 c\\
                4^3 = 4 a + 6 b + 4 a\\
                5^3 = 10 a + 10 a + 5 c
            \end{cases}
        \]
        解得, $a = 6$ , $b = -6$ , $c= 1$ .
        
        容易验证对 $n = 1, 2$ , 命题也成立. 假设这个命题对 $m-1$ 成立, 则对 $m$
        \begin{equation*}
            \begin{split}
                & \quad 6{m \choose 3} -6 {m \choose 2} + {m \choose 1}\\
                & = 6{m - 1 \choose 3} -6 {m - 1 \choose 2} + {m - 1 \choose 1} + 6{m - 1 \choose 2} -6 {m - 1 \choose 1} + {m - 1 \choose 0}\\
                & = (m - 1)^3 + 3 m^2 - 9m + 6 + 6m -6 + 1 = m^3
            \end{split}
        \end{equation*}
        故命题对任意正整数都成立.

        那么求和式的结果为
        \begin{equation*}
            \begin{split}
                \sum^{n}_{m=1} m^3 
                & = 6\sum^{n}_{m=3}{m \choose 3} -6\sum^{n}_{m=3}{m \choose 2} + \sum^{n}_{m=1}{m \choose 1}\\
                & = 6 {n + 1 \choose 4} -6 {n+1 \choose 3} + {n+1 \choose 2}\\
                & = \frac{1}{4} n^2(n-1)^2
            \end{split}
        \end{equation*}
    \paragraph{3.7.2}\quad 
        若 $k \ge \lceil{\frac{n}{2}}\rceil \, (\lceil{*}\rceil\text{表示下取整})$ , ${{n} \choose {k+1}} - {{n} \choose {k}} < 0$ , 那么取到最大值的 $k$ 一定小于 $\lceil{\frac{n}{2}}\rceil$ . 
        \[
            \begin{split}
                &\quad\biggl({{n} \choose {k+1}} - {{n} \choose {k}}\biggl) 
                - \biggl({{n} \choose {k}} - {{n} \choose {k-1}}\biggl)\\
                & = {{n} \choose {k+1}} - 2{{n} \choose {k}} + {{n} \choose {k - 1}}\\
                & = (\frac{k}{n-k+1} + \frac{n-k}{k+1} -2) {{n} \choose {k}}\\
                & = \frac{-(n-2 k)^2+n+2}{(k+1) (k-n-1)} {{n} \choose {k}} \ge 0
            \end{split}
        \]
        由 $k<\lceil{\frac{n}{2}}\rceil$ , 可得 $k \le \frac{n - \sqrt{n+2}}{2}$ , 考虑到 $k \in \mathbb{N}$ , $k = [\frac{n - \sqrt{n+2}}{2}]\,([*]\text{表示取整, 四舍五入})$ 时, ${{n} \choose {k}} - {{n} \choose {k-1}}$ 取到最大值.
    \paragraph{补充1}\quad 
        \begin{equation*}
            \begin{split}
                \sum^{n}_{k=1} k{n \choose k}^2
                & = \frac{1}{2} \sum^{n}_{k=1} k{n \choose k}^2 + \frac{1}{2} \sum^{n}_{k=1} k{n \choose k}^2\\
                & = \frac{1}{2} \sum^{n}_{k=1} k{n \choose k}^2 + \frac{1}{2} \sum^{n}_{k=1} k{n \choose n - k}^2\\
                & = \frac{n}{2} \sum^{n}_{k=1} {n \choose k}^2\\
                & = \frac{n}{2} {2n \choose n}\\
                & = n {2n-1 \choose n-1}
            \end{split}
        \end{equation*}
    \paragraph{补充2}\quad 
        由多项式定理 
        \[
            \bigg(\sum^{t}_{k=1} x_k\bigg)^n = \sum^{}_{} {n \choose n_1 \cdots n_t}x_1^{n_1} \cdots x_t^{n_t}
        \]
        令 $x_1 = \cdots = x_t = 1$ , 则可得
        $$t^n = \sum^{}_{} {n \choose n_1 \cdots n_t}$$
    \paragraph{补充3}\quad
        \begin{equation*}
            \begin{split}
                {\frac{1}{3} \choose k}
                & = \frac{\frac{1}{3}(\frac{1}{3}-1)\cdots(\frac{1}{3}-k+1)}{k!}\\
                & = (-1)^{k-1}\frac{2 \times 5 \times \cdots \times 3k - 1}{3^k k!}
            \end{split}
        \end{equation*}
        则可以求得 ${\frac{1}{3} \choose 0} = 0$ , ${\frac{1}{3} \choose 1} = 1/3$ , ${\frac{1}{3} \choose 2} = -1/9$ , ${\frac{1}{3} \choose 3} = 5/58$
        \[
            10^{\frac{1}{3}} = 8^{\frac{1}{3}}(1+0.25)^{\frac{1}{3}} = 2\sum^{\infty}_{k=0} {\frac{1}{3} \choose k} 0.25^k \approx 0 + \frac{2}{3} 0.25 - \frac{2}{9} 0.25^2 + 2\frac{10}{58} 0.25^3 = 2.15439
        \]
    \paragraph{补充4}\quad 
        \begin{equation*}
            \begin{split}
                \sum^{n}_{k=0}(-1)^k \frac{1}{k+1} {n \choose k}
                & = \frac{1}{n+1} \sum^{n}_{k=0}(-1)^k {n + 1 \choose k + 1}\\
                & = \frac{1}{n+1} \sum^{n+1}_{k=0}(-1)^{k - 1} {n + 1 \choose k} + \frac{1}{n+1}\\
                & = -\frac{1}{n+1}(1-1)^{n+1} + \frac{1}{n+1}\\
                & = \frac{1}{n+1}
            \end{split}
        \end{equation*}
\end{document}										%结束正文
